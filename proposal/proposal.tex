\documentclass{article}

% if you need to pass options to natbib, use, e.g.:
% \PassOptionsToPackage{numbers, compress}{natbib}
% before loading nips_2016
%
% to avoid loading the natbib package, add option nonatbib:
% \usepackage[nonatbib]{nips_2016}

%\usepackage{nips_2016}

% to compile a camera-ready version, add the [final] option, e.g.:
\usepackage[final]{nips_2016}

\usepackage[utf8]{inputenc} % allow utf-8 input
\usepackage[T1]{fontenc}    % use 8-bit T1 fonts
%\usepackage{hyperref}       % hyperlinks
\usepackage{url}            % simple URL typesetting
\usepackage{booktabs}       % professional-quality tables
\usepackage{amsfonts}       % blackboard math symbols
\usepackage{nicefrac}       % compact symbols for 1/2, etc.
\usepackage{microtype}      % microtypography

\title{Twitter Sentiment Analysis of the Brexit result}

% The \author macro works with any number of authors. There are two
% commands used to separate the names and addresses of multiple
% authors: \And and \AND.
%
% Using \And between authors leaves it to LaTeX to determine where to
% break the lines. Using \AND forces a line break at that point. So,
% if LaTeX puts 3 of 4 authors names on the first line, and the last
% on the second line, try using \AND instead of \And before the third
% author name.

\author{
    Surya Vajjhala \\
    U19590925\\
    \texttt{Email} vajjhala@bu.edu\\
     \And
    Zhi Dou \\
    U21392913 \\
    \texttt{Email} zhidou@bu.edu\\
}

\begin{document}
% \nipsfinalcopy is no longer used

\maketitle

\section*{Nature of the Dataset}
We dataset are tweets of twitter, from Jan 1, 2016 to Oct 1, 2016

\subsection*{\textit{\textbf{Collection:}}}
We scrape data with tag \#Brexit using Twitter API 
\subsection*{\textit{\textbf{Format:}}} Twitter API provides the data in JSON format. These data also called tweets are the basic atomic building block of all things in Twitter. It contains keys like: \lq text\rq, \lq retweeted\_status\rq, \lq favorite\_count\rq, \lq metadata\rq, \lq user\rq, \lq created\_at\rq, \lq source\rq, \lq retweeted\rq, \lq coordinates\rq, \lq possibly\_sensitive\rq, \lq truncated\rq, \lq contributors\rq, \lq is\_quote\_status\rq, \lq id\rq, \lq retweet\_count\rq, \lq lang\rq, \lq place\rq , \lq entities\rq , \lq favorited\rq	

\par For our project, the keys we may need are \lq Text\rq, \lq location\rq, \lq language\rq, \lq tweets created time\rq.

\subsection*{\textit{\textbf{Preprocess:}}} Because the data got from API has more information than we need, we should parse out the raw data extract the attributes we would use and save it into cvs file for later usage.


\section*{Expected Analysis}

\subsection* {\textit{\textbf{Latent Semantic Analysis}}}

\par In order to get information from text of each tweets, we should do latent semantic analysis on the texts we get, to break texts into words, to remove meaningless high frequency words, and to reduce the dimension we may need for further step.

\subsection*{\textit{\textbf{Sentiment Analysis}}}
\par To get the attitude of each user towards Brexit, we should implement sentiment analysis on the data we have. 

\subsection*{\textit{\textbf{Clusterring}}}
\par For mining the relationship between each data, we would cluster data the opinion and location of each data. We may use Kmeans and GMM to do clustering.

\section*{Application}
\par Based on our work, we could analysis the trends in the Brexit vote. We could answer questions like below:
\begin{enumerate}
\item Prediction of the Brexit vote based on twitter data.
\item Does UK regret the Brexit?
\item General opinion of others nations (more specifically other countries of EU) towards the result of the referendum.
\item A timeline of variation of public opinion leading up to the vote.
\item Demographic insights of the Brexit vote.
\end{enumerate}

We could get the answer of most of this question, after we finish sentiment analysis of the data we get. And through our analysis, it is easy to see the big picture and trends on Brexit.

\section*{Expected Results}

\begin{enumerate}
\item We might validate Twitter could correctly predict the Brexit vote result.
\item We expect that there is a large population that regrets the result of the voting.
\item We cannot make assumptions until we analyses tweets from each country.
\item Show a plot of changing variation and try to show the changes actually related to major socio-political events ( like the immigration crisis, bombing in Pari, 2015 UK general election, greek bailout referendum, etc.).
\item Remain camps clustered around major cities and Leave camps around the rural and country side.

\end{enumerate}

\end{document}